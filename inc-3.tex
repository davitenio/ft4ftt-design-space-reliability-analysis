\documentclass[times,a4paper,10pt,twocolumn]{article}
\usepackage{times}

\usepackage{amssymb}
\usepackage{amsmath}
\usepackage{amsthm}

\usepackage[scientific-notation=true]{siunitx}

\usepackage{sagetex}

\usepackage{datetime}

\usepackage{subfig}

\usepackage{calc}
\usepackage{ctable}

\theoremstyle{definition}
\newtheorem{definition}{Definition}[section]

\theoremstyle{definition}
\newtheorem{assumption}{Assumption}[section]

\theoremstyle{plain}
\newtheorem{proposition}{Proposition}[section]




\begin{document}

\begin{sagesilent}
import reliabilityanalysis.prismmodel as pm
import datetime
import itertools

# number of time units that the PRISM experiments should last
EXPERIMENT_DURATION = 20

def get_chunks(iterable, size):
    i = iter(iterable)
    chunk = list(itertools.islice(i, size))
    while chunk:
        yield chunk
        chunk = list(itertools.islice(i, size))

def generate_latex_figures(
        plots_generator,
        # list of numbers with which to label each of the plots
        subfloat_numbers,
        fig_name_prefix,
        caption,
        label,
        # number of columns in each figure
        numcolumns = 5,
        # number of rows in each figure
        numrows = 8):
    plots_per_fig = numcolumns * numrows
    plot_index = 0
    figure_index = 0
    latex_strings = []
    for chunk in get_chunks(plots_generator, plots_per_fig):
        latex_strings.append(r'\begin{figure*}')
        latex_strings.append(r'\renewcommand*\thesubfigure{\arabic{subfigure}}')
        if figure_index != 0:
            latex_strings.append(r'\ContinuedFloat')
        latex_strings.append(r'\centering')
        for p in chunk:
            latex_strings.append(r'\setcounter{subfigure}{'+
                str(subfloat_numbers[plot_index] - 1) + '}')
            plot_file_name = 'sage-plots-for-inc-3.tex/' + \
                fig_name_prefix + str(plot_index) + '.pdf'
            p.save(filename=plot_file_name)
            latex_strings.append(r'\subfloat[]{\includegraphics[width=' +
                str(n(1/numcolumns)) + r'\textwidth-0.5em, height=' +
                str(n(1/numrows)) + r'\textheight-3em, keepaspectratio]{' +
                plot_file_name + '}}')
            plot_index += 1
            if plot_index % numcolumns == 0:
                latex_strings.append(r'\\')
        figure_index += 1
        latex_strings.append(r'\caption{' + caption +
            r' \emph{(cont.~page~\pageref{' +
            label + str(figure_index + 1) + r'})}}')
        latex_strings.append(r'\label{' + label + str(figure_index) + '}')
        latex_strings.append(r'\end{figure*}')
        latex_strings.append(r'\clearpage')
    latex_strings[-4] = r'\caption{' + caption + '}'
    latex_figure = '\n'.join(latex_strings)
    return latex_figure

def graph_plot_generator(
        graphs,
        graphs_to_highlight = [],
        layout="circular",
        vertex_size=12000,
        font_size=70,
        edge_thickness=10,
        border_offset=0.5,
        border_style_default=":",
        border_style_highlight="-",
        border_thickness=10):
    """
    We generate the graph plots ourselves using scatterplots because they allow
    us to better control how the plots will look.
    """
    for g in graphs:
        # get vertex positions
        g.plot(layout=layout, save_pos=True)
        vertex_positions = [g.get_pos()[v] for v in g.vertices()]
        # plot vertices
        graph_plot = scatter_plot(vertex_positions, markersize=vertex_size,
            axes=False, facecolor="white")
        vertex_labels = [text(str(v), g.get_pos()[v], fontsize=font_size,
            color="black", zorder=10) for v in g.vertices()]
        graph_plot += sum(vertex_labels)
        # plot edges
        edge_plots = []
        for v in g.vertices():
            for n in g.neighbors(v):
                edge_plots.append(line((g.get_pos()[v], g.get_pos()[n]),
                    color="black", thickness=edge_thickness))
        graph_plot += sum(edge_plots)
        # plot graph border
        if g.order() > 1:
            xmin = min([g.get_pos()[v][0] for v in g.vertices()])
            ymin = min([g.get_pos()[v][1] for v in g.vertices()])
            xmax = max([g.get_pos()[v][0] for v in g.vertices()])
            ymax = max([g.get_pos()[v][1] for v in g.vertices()])
        else:
            xmin = -1
            ymin = -1
            xmax =  1
            ymax =  1
        xmin, ymin = map(lambda x: x - border_offset, [xmin, ymin])
        xmax, ymax = map(lambda x: x + border_offset, [xmax, ymax])
        if g in graphs_to_highlight:
            border_style = border_style_highlight
        else:
            border_style = border_style_default
        border_plot = line([(xmin, ymin), (xmin, ymax),
            (xmax, ymax), (xmax, ymin), (xmin, ymin)],
            linestyle=border_style, thickness=border_thickness)
        graph_plot += border_plot
        yield graph_plot

def induced_subdigraph_plot_generator(
        digraph,
        vertex_size=4000,
        font_size=30,
        border_offset=0.5,
        border_style="-",
        border_thickness=5):
    induced_subdigraphs = []
    for v in digraph.vertices():
        dg = digraph.subgraph(vertices=[v] + digraph.neighbors_out(v))
        assert dg.is_connected()
        induced_subdigraphs.append(dg)

    for subdigraph in induced_subdigraphs:
        subdigraph_plot = subdigraph.plot(layout="circular",
            vertex_size=vertex_size, vertex_labels=False, save_pos=True)
        vertex_labels = [text(str(v), subdigraph.get_pos()[v],
            fontsize=font_size, color="black", zorder=10,
            axes=False) for v in subdigraph.vertices()]
        subdigraph_plot += sum(vertex_labels)
        if subdigraph.order() > 2:
            xmin = min([subdigraph.get_pos()[v][0] for v in
                subdigraph.vertices()])
            ymin = min([subdigraph.get_pos()[v][1] for v in
                subdigraph.vertices()])
            xmax = max([subdigraph.get_pos()[v][0] for v in
                subdigraph.vertices()])
            ymax = max([subdigraph.get_pos()[v][1] for v in
                subdigraph.vertices()])
        else:
            xmin = -1
            ymin = -1
            xmax =  1
            ymax =  1
        xmin, ymin = map(lambda x: x - border_offset, [xmin, ymin])
        xmax, ymax = map(lambda x: x + border_offset, [xmax, ymax])
        border_plot = line([(xmin, ymin), (xmin, ymax),
            (xmax, ymax), (xmax, ymin), (xmin, ymin)],
            linestyle=border_style, thickness=border_thickness)
        subdigraph_plot += border_plot
        yield subdigraph_plot

def generate_latex_table(d, caption, label):
    row_strings = [r"%s & \num{%s} \NN" % (k, v)
        for (k, v) in d.items()]
    row_strings[-1] = row_strings[-1].rstrip(r" \NN")
    table_content = " ".join(row_strings)
    template = """
        \ctable[
        caption = {%s},
        label = {%s},
        ]{lr}{
        }{
        \FL
            Component  & Failure rate
        \ML
            %s
        \LL
        }
        """ % (caption, label, table_content)

    return template.replace("\n", "")

\end{sagesilent}

\title{
Reliability analysis of design spaces based on Markov chains derived from a
graph model assuming perfect coverage
}

\author{David Gessner\\
Dpt. Matem\`atiques i Inform\`atica, Universitat de les Illes Balears, Spain\\
davidges@gmail.com\
\date{\today~at {\currenttime}h}
}

\maketitle
\thispagestyle{empty}





\begin{abstract}

\end{abstract}







\section{Introduction}

\section{Examples}
\label{sec:Examples}


\subsection{Simplex switch}

In this example we consider a simplex star topology, i.e., a single switch to
which several network stations are connected. Figure~\ref{fig:simplex-switch}
shows the graph modeling this system.

%\sagetexpause

\begin{sagesilent}

SLAVE_COUNT = 3

switch = "$w_{}$"
slaves = ["$s_{}$".format(i) for i in range(1, SLAVE_COUNT + 1)]
links = ["$l_{}$".format(i) for i in range(1, SLAVE_COUNT + 1)]
simplex_switch = Graph()
simplex_switch.add_edges(zip(slaves, links))
simplex_switch.add_edges([(l, switch) for l in links])

plot_generator = graph_plot_generator([simplex_switch], vertex_size=6000,
    font_size=35, edge_thickness=5, border_style_default="")
\end{sagesilent}

\begin{figure}
\centering
\sageplot[width=4cm][pdf]{
next(plot_generator)
}
\caption{Graph corresponding to the simplex switch example.}
\label{fig:simplex-switch}
\end{figure}

\sagetexpause

\begin{sagesilent}

def exists_slave_connected_to_switch(graph):
    if ( ((slaves[0], switches[0], None) in graph.edges()) or
            ((slaves[1], switches[0], None) in graph.edges()) ):
        return True
    return False

SIMPLEX_SWITCH_EDGE_FAILURE_RATE = 1/100
SIMPLEX_SWITCH_VERTEX_FAILURE_RATE = 2/100

edge_failure_rates = {edge[:2]: SIMPLEX_SWITCH_EDGE_FAILURE_RATE
    for edge in simplex_switch.edges()}
vertex_failure_rates = {vertex: SIMPLEX_SWITCH_VERTEX_FAILURE_RATE
    for vertex in simplex_switch.vertices()}
failure_rates = dict(edge_failure_rates.items() + vertex_failure_rates.items())

digraph, transition_rate_matrix = pm.generate_ctmc(simplex_switch,
    failure_rates)
transitions_file_name, states_file_name = pm.create_prism_files(
    transition_rate_matrix, 'prism/simplex_star')
non_faulty_states = pm.filter_graphs(digraph,
    exists_slave_connected_to_switch)
pm.prism_properties(non_faulty_states, 'prism/simplex_star.pctl')

now_str = datetime.datetime.now().isoformat()
results_dir_name = "results/simplex_star_{}/".format(now_str)
os.makedirs(results_dir_name)
pm.run_experiments(transitions_file_name, states_file_name,
    'prism/simplex_star.pctl', results_dir_name,
    non_faulty_states, EXPERIMENT_DURATION)

results_plot = pm.plot_results(results_dir_name, EXPERIMENT_DURATION)

subgraphs = []
vertex_numbers = []
non_faulty_subgraphs = []
for v in digraph.vertices():
    subgraphs.append(digraph.get_vertex(v))
    vertex_numbers.append(v)
    if v in non_faulty_states:
        non_faulty_subgraphs.append(digraph.get_vertex(v))

subgraph_vertex_plots = graph_plot_generator(subgraphs,
    graphs_to_highlight=non_faulty_subgraphs)


\end{sagesilent}

Figure~\ref{fig:simplex-switch-digraph} shows the digraph corresponding to the
simplex switch example. Each of the vertices of this digraph corresponds to a
subgraph of the graph shown in Figure~\ref{fig:simplex-switch}.
Figure~\ref{fig:simplex_switch_subgraphs_1} shows which specific subgraphs
correspond to each vertex of the digraph. Vertices of the digraph are
considered non-faulty if they have at least one station connected to the
switch. These vertices are highlighted in
Figure~\ref{fig:simplex_switch_subgraphs_1} with a solid border.

Figure~\ref{fig:simplex-switch-results} shows the reliability of each one of
the non-faulty vertices of the digraph assuming we have a failure rate of
$\sage{SIMPLEX_SWITCH_VERTEX_FAILURE_RATE}$ for all stations and the switch,
and a failure rate of $\sage{SIMPLEX_SWITCH_EDGE_FAILURE_RATE}$ for all links
connecting stations with the switch.



\begin{figure}
\centering
\sageplot[width=\columnwidth][pdf]{
digraph.plot(layout="circular")
}
\caption{Digraph corresponding to the simplex switch.}
\label{fig:simplex-switch-digraph}
\end{figure}

\sagestr{generate_latex_figures(subgraph_vertex_plots, vertex_numbers,
    "simplex_switch_vertices_",
    'Vertices of the digraph corresponding to the simplex switch.',
    'fig:simplex_switch_subgraphs_')}

\begin{figure}
\centering
\sageplot[width=\columnwidth][pdf]{
results_plot
}
\caption{Results for the simplex switch.}
\label{fig:simplex-switch-results}
\end{figure}








\section*{Acknowledgements}

This work was supported by project DPI2011-22992 and grant BES-2012-052040
(Spanish \emph{Ministerio de econom\'ia y competividad}), and by FEDER funding.






\bibliographystyle{abbrv}
\bibliography{inc-3}


\end{document}

% vim: tabstop=8 expandtab shiftwidth=4 softtabstop=4
