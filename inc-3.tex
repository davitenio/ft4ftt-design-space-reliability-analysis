\documentclass[times,a4paper,10pt,twocolumn]{article}
\usepackage{times}

\usepackage{amssymb}
\usepackage{amsmath}
\usepackage{amsthm}

\usepackage[scientific-notation=true]{siunitx}

\usepackage{sagetex}

\usepackage{datetime}

\usepackage{subfig}

\usepackage{calc}
\usepackage{ctable}

\theoremstyle{definition}
\newtheorem{definition}{Definition}[section]

\theoremstyle{definition}
\newtheorem{assumption}{Assumption}[section]

\theoremstyle{plain}
\newtheorem{proposition}{Proposition}[section]




\begin{document}

\begin{sagesilent}
import reliabilityanalysis.prismmodel as pm
import datetime
import itertools

# number of time units that the PRISM experiments should last
EXPERIMENT_DURATION = 20

def get_chunks(iterable, size):
    i = iter(iterable)
    chunk = list(itertools.islice(i, size))
    while chunk:
        yield chunk
        chunk = list(itertools.islice(i, size))

def generate_latex_figures(
        plots_generator,
        # list of numbers with which to label each of the plots
        subfloat_numbers,
        fig_name_prefix,
        caption,
        label,
        # number of columns in each figure
        numcolumns = 5,
        # number of rows in each figure
        numrows = 8):
    plots_per_fig = numcolumns * numrows
    plot_index = 0
    figure_index = 0
    latex_strings = []
    for chunk in get_chunks(plots_generator, plots_per_fig):
        latex_strings.append(r'\begin{figure*}')
        latex_strings.append(r'\renewcommand*\thesubfigure{\arabic{subfigure}}')
        if figure_index != 0:
            latex_strings.append(r'\ContinuedFloat')
        latex_strings.append(r'\centering')
        for p in chunk:
            latex_strings.append(r'\setcounter{subfigure}{'+
                str(subfloat_numbers[plot_index] - 1) + '}')
            plot_file_name = 'sage-plots-for-inc-3.tex/' + \
                fig_name_prefix + str(plot_index) + '.pdf'
            p.save(filename=plot_file_name)
            latex_strings.append(r'\subfloat[]{\includegraphics[width=' +
                str(n(1/numcolumns)) + r'\textwidth-0.5em, height=' +
                str(n(1/numrows)) + r'\textheight-3em, keepaspectratio]{' +
                plot_file_name + '}}')
            plot_index += 1
            if plot_index % numcolumns == 0:
                latex_strings.append(r'\\')
        figure_index += 1
        latex_strings.append(r'\caption{' + caption +
            r' \emph{(cont.~page~\pageref{' +
            label + str(figure_index + 1) + r'})}}')
        latex_strings.append(r'\label{' + label + str(figure_index) + '}')
        latex_strings.append(r'\end{figure*}')
        latex_strings.append(r'\clearpage')
    latex_strings[-4] = r'\caption{' + caption + '}'
    latex_figure = '\n'.join(latex_strings)
    return latex_figure

def graph_plot_generator(
        graphs,
        graphs_to_highlight = [],
        layout="circular",
        vertex_size=12000,
        font_size=70,
        edge_thickness=10,
        border_offset=0.5,
        border_style_default=":",
        border_style_highlight="-",
        border_thickness=10):
    """
    We generate the graph plots ourselves using scatterplots because they allow
    us to better control how the plots will look.
    """
    for g in graphs:
        # get vertex positions
        g.plot(layout=layout, save_pos=True)
        vertex_positions = [g.get_pos()[v] for v in g.vertices()]
        # plot vertices
        graph_plot = scatter_plot(vertex_positions, markersize=vertex_size,
            axes=False, facecolor="white")
        vertex_labels = [text(str(v), g.get_pos()[v], fontsize=font_size,
            color="black", zorder=10) for v in g.vertices()]
        graph_plot += sum(vertex_labels)
        # plot edges
        edge_plots = []
        for v in g.vertices():
            for n in g.neighbors(v):
                edge_plots.append(line((g.get_pos()[v], g.get_pos()[n]),
                    color="black", thickness=edge_thickness))
        graph_plot += sum(edge_plots)
        # plot graph border
        if g.order() > 1:
            xmin = min([g.get_pos()[v][0] for v in g.vertices()])
            ymin = min([g.get_pos()[v][1] for v in g.vertices()])
            xmax = max([g.get_pos()[v][0] for v in g.vertices()])
            ymax = max([g.get_pos()[v][1] for v in g.vertices()])
        else:
            xmin = -1
            ymin = -1
            xmax =  1
            ymax =  1
        xmin, ymin = map(lambda x: x - border_offset, [xmin, ymin])
        xmax, ymax = map(lambda x: x + border_offset, [xmax, ymax])
        if g in graphs_to_highlight:
            border_style = border_style_highlight
        else:
            border_style = border_style_default
        border_plot = line([(xmin, ymin), (xmin, ymax),
            (xmax, ymax), (xmax, ymin), (xmin, ymin)],
            linestyle=border_style, thickness=border_thickness)
        graph_plot += border_plot
        yield graph_plot

def induced_subdigraph_plot_generator(
        digraph,
        vertex_size=4000,
        font_size=30,
        border_offset=0.5,
        border_style="-",
        border_thickness=5):
    induced_subdigraphs = []
    for v in digraph.vertices():
        dg = digraph.subgraph(vertices=[v] + digraph.neighbors_out(v))
        assert dg.is_connected()
        induced_subdigraphs.append(dg)

    for subdigraph in induced_subdigraphs:
        subdigraph_plot = subdigraph.plot(layout="circular",
            vertex_size=vertex_size, vertex_labels=False, save_pos=True)
        vertex_labels = [text(str(v), subdigraph.get_pos()[v],
            fontsize=font_size, color="black", zorder=10,
            axes=False) for v in subdigraph.vertices()]
        subdigraph_plot += sum(vertex_labels)
        if subdigraph.order() > 2:
            xmin = min([subdigraph.get_pos()[v][0] for v in
                subdigraph.vertices()])
            ymin = min([subdigraph.get_pos()[v][1] for v in
                subdigraph.vertices()])
            xmax = max([subdigraph.get_pos()[v][0] for v in
                subdigraph.vertices()])
            ymax = max([subdigraph.get_pos()[v][1] for v in
                subdigraph.vertices()])
        else:
            xmin = -1
            ymin = -1
            xmax =  1
            ymax =  1
        xmin, ymin = map(lambda x: x - border_offset, [xmin, ymin])
        xmax, ymax = map(lambda x: x + border_offset, [xmax, ymax])
        border_plot = line([(xmin, ymin), (xmin, ymax),
            (xmax, ymax), (xmax, ymin), (xmin, ymin)],
            linestyle=border_style, thickness=border_thickness)
        subdigraph_plot += border_plot
        yield subdigraph_plot

def generate_latex_table(d, caption, label):
    row_strings = [r"%s & \num{%s} \NN" % (k, v)
        for (k, v) in d.items()]
    row_strings[-1] = row_strings[-1].rstrip(r" \NN")
    table_content = " ".join(row_strings)
    template = """
        \ctable[
        caption = {%s},
        label = {%s},
        ]{lr}{
        }{
        \FL
            Component  & Failure rate
        \ML
            %s
        \LL
        }
        """ % (caption, label, table_content)

    return template.replace("\n", "")

\end{sagesilent}

\title{
Reliability analysis of design spaces based on Markov chains derived from a
graph model assuming perfect coverage
}

\author{David Gessner\\
Dpt. Matem\`atiques i Inform\`atica, Universitat de les Illes Balears, Spain\\
davidges@gmail.com\
\date{\today~at {\currenttime}h}
}

\maketitle
\thispagestyle{empty}





\begin{abstract}

\end{abstract}







\section{Introduction}

When designing a system to solve a problem there are usually many different
design decisions that need to be made. Depending on the choice made for each of
these decisions, we may come up with one design or another. If there are $n$
design decisions that need to be made, the set of possible designs can be
thought of as an $n$-dimensional space, where each dimension corresponds to one
particular design decision. The values for a dimension are the possible
alternatives for the corresponding design decision and each point of the space
is a possible design that solves the problem. This space is known as the
\emph{design space} for the given problem. In this paper we will only consider
design spaces with a finite number of points.

Note that the dimensions of a design space do not need to be orthogonal since
some design decisions may constrain others.

It is often desirable to find the optimal design, according to some criteria,
that solves a given problem. In this paper the main criteria for making a
choice among the possible designs is the reliability of each design.

The paper presents a method to evaluate the reliability of all points of a
design space by creating a single mathematical model from which models for all
points of the design space can be obtained. The modeling formalism that we will
use are undirected graphs together with a Boolean function expressed in terms
of graph properties (such as connectivity, for example) that allows us to
distinguish between when the modeled system should be declared as faulty and
when it should be declared as non-faulty. By evaluating the reliability of all
points of a design space we are guaranteed to find the optimal solution in
terms of reliability.



\section{Basic concepts}

\begin{definition}

A \emph{(undirected) graph} is a tuple
\[
G = (V, E)
\]
where $V$ is a set of arbitrary mathematical objects whose elements are called
the \emph{vertices} of the graph and $E \subseteq \{\{u, v\} \mid u,v \in V\}$
is a set of unordered vertex pairs called \emph{edges}.

\end{definition}


\begin{definition}

A \emph{directed graph} or \emph{digraph} is a tuple
\[
D = (V, E)
\]
where $V$ is a set of arbitrary mathematical objects whose elements are called
the \emph{vertices} of the digraph and $E \subseteq \{(u, v)\mid (u, v) \in V
\times V\}$ is a set of ordered pairs of vertices called \emph{(directed)
edges}.

\end{definition}

\begin{definition}

A graph $G_2 = (V_2, E_2)$ is a \emph{subgraph} of a graph $G_1 = (V_1, E_1)$
if $V_2 \subseteq V_1$ and $E_2 \subseteq E_1$.

\end{definition}

\begin{definition}

A digraph $D_2 = (V_2, E_2)$ is a \emph{subdigraph} of a digraph $D_1 = (V_1,
E_1)$ if $V_2 \subseteq V_1$ and $E_2 \subseteq E_1$.

\end{definition}


\begin{definition}

The \emph{union} of two graphs $G_1 = (V_1, E_1)$ and $G_2 = (V_2, E_2)$ is a
graph $G_3 = (V_1 \cup V_2, E_1 \cup E_2)$. We write $G_3 = G_1 \cup G_2$.

\end{definition}





\section{Modeling of design spaces}

In our approach, we assume that each point of the design space can be
represented by a graph.

\begin{definition}

Let $S = \{G_1, G_2, \ldots, G_N\}$ be the set of points of a design space. The
\emph{design space union graph} is the graph
\[
G_S = \bigcup \limits_{i = 1}^{N} G_i
\]

\end{definition}



\begin{definition}

Let $A(G_S)$ be the set of also all subgraphs of $G_S$. The \emph{extended
design space} is the design space whose points are the elements of $A(G_S)$.

\end{definition}

Note that $S \subseteq A(G_S)$.

\begin{definition}

The \emph{extended design space digraph} is a digraph whose vertices are the
points of the extended design space corresponding to $A(G_S)$. Note that this
digraph is a digraph whose vertices are undirected graphs. The directed edges
of this digraph are the set $\{(G_i, G_j) : G_i \neq G_j, G_j \subseteq G_i,
\lvert V(G_j) \setminus V(G_i) \rvert \leq 1, \lvert E(G_j) \setminus E(G_i)
\rvert \leq 1\}$. In other words, in the digraph there is a directed edge $(u,
v)$ iff $v$ is a proper subgraph of $u$ and $v$ can be obtained from $u$ by
means of a single edge or vertex deletion.

\end{definition}





\section{Deriving a Markov chain from the modeled design space}





\section{Examples}
\label{sec:Examples}


\subsection{Simplex switch}

In this example we consider a simplex star topology, i.e., a single switch to
which several network stations are connected. Figure~\ref{fig:simplex-switch}
shows the graph modeling this system.

%\sagetexpause

\begin{sagesilent}

SLAVE_COUNT = 2
SWITCH_COUNT = 1

slaves = ["$s_{}$".format(i) for i in range(1, SLAVE_COUNT + 1)]
switches = ["$w_{}$".format(i) for i in range(1, SWITCH_COUNT + 1)]
simplex_switch = Graph()
simplex_switch.add_edges([(s, w) for s in slaves for w in switches])

plot_generator = graph_plot_generator([simplex_switch], vertex_size=6000,
    font_size=35, edge_thickness=5, border_style_default="")
\end{sagesilent}

\begin{figure}
\centering
\sageplot[width=4cm][pdf]{
next(plot_generator)
}
\caption{Graph corresponding to the simplex switch example.}
\label{fig:simplex-switch}
\end{figure}

\begin{sagesilent}

def exists_slave_connected_to_switch(graph):
    if ( ((slaves[0], switches[0], None) in graph.edges()) or
            ((slaves[1], switches[0], None) in graph.edges()) ):
        return True
    return False

SIMPLEX_SWITCH_EDGE_FAILURE_RATE = 1/100
SIMPLEX_SWITCH_VERTEX_FAILURE_RATE = 2/100

edge_failure_rates = {edge[:2]: SIMPLEX_SWITCH_EDGE_FAILURE_RATE
    for edge in simplex_switch.edges()}
vertex_failure_rates = {vertex: SIMPLEX_SWITCH_VERTEX_FAILURE_RATE
    for vertex in simplex_switch.vertices()}
failure_rates = dict(edge_failure_rates.items() + vertex_failure_rates.items())

digraph, transition_rate_matrix = pm.generate_ctmc(simplex_switch,
    failure_rates)
transitions_file_name, states_file_name = pm.create_prism_files(
    transition_rate_matrix, 'prism/simplex_star')
non_faulty_states = pm.filter_graphs(digraph,
    exists_slave_connected_to_switch)
pm.prism_properties(non_faulty_states, 'prism/simplex_star.pctl')

now_str = datetime.datetime.now().isoformat()
results_dir_name = "results/simplex_star_{}/".format(now_str)
os.makedirs(results_dir_name)
pm.run_experiments(transitions_file_name, states_file_name,
    'prism/simplex_star.pctl', results_dir_name,
    non_faulty_states, EXPERIMENT_DURATION)

results_plot = pm.plot_results(results_dir_name, EXPERIMENT_DURATION)

subgraphs = []
vertex_numbers = []
non_faulty_subgraphs = []
for v in digraph.vertices():
    subgraphs.append(digraph.get_vertex(v))
    vertex_numbers.append(v)
    if v in non_faulty_states:
        non_faulty_subgraphs.append(digraph.get_vertex(v))

subgraph_vertex_plots = graph_plot_generator(subgraphs,
    graphs_to_highlight=non_faulty_subgraphs)


\end{sagesilent}

Figure~\ref{fig:simplex-switch-digraph} shows the digraph corresponding to the
simplex switch example. Each of the vertices of this digraph corresponds to a
subgraph of the graph shown in Figure~\ref{fig:simplex-switch}.
Figure~\ref{fig:simplex_switch_subgraphs_1} shows which specific subgraphs
correspond to each vertex of the digraph. Vertices of the digraph are
considered non-faulty if they have at least one station connected to the
switch. These vertices are highlighted in
Figure~\ref{fig:simplex_switch_subgraphs_1} with a solid border.

Figure~\ref{fig:simplex-switch-results} shows the reliability of each one of
the non-faulty vertices of the digraph assuming we have a failure rate of
$\sage{SIMPLEX_SWITCH_VERTEX_FAILURE_RATE}$ for all stations and the switch,
and a failure rate of $\sage{SIMPLEX_SWITCH_EDGE_FAILURE_RATE}$ for all links
connecting stations with the switch.



\begin{figure}
\centering
\sageplot[width=\columnwidth][pdf]{
digraph.plot(layout="circular")
}
\caption{Digraph corresponding to the simplex switch.}
\label{fig:simplex-switch-digraph}
\end{figure}

\sagestr{generate_latex_figures(subgraph_vertex_plots, vertex_numbers,
    "simplex_switch_vertices_",
    'Vertices of the digraph corresponding to the simplex switch.',
    'fig:simplex_switch_subgraphs_')}

\begin{figure}
\centering
\sageplot[width=\columnwidth][pdf]{
results_plot
}
\caption{Results for the simplex switch.}
\label{fig:simplex-switch-results}
\end{figure}




%\sagetexpause

\subsection{Replicated switch}

In this example we will consider a replicated star topology comprised of two
interconnected switches and two network stations, each connected to both
switches. The topology is illustrated in Figure~\ref{fig:replicated-switch}.
The stations are labeled as $s_1$ and $s_2$; whereas the switches are labeled
as $w_1$ and $w_2$.

\begin{sagesilent}

SLAVE_COUNT = 2
SWITCH_COUNT = 2

slaves = ["$s_{}$".format(i) for i in range(1, SLAVE_COUNT + 1)]
switches = ["$w_{}$".format(i) for i in range(1, SWITCH_COUNT + 1)]
replicated_switch = Graph()
replicated_switch.add_edges([(s, w) for s in slaves for w in switches])
replicated_switch.add_edges([(w1, w2) for w1 in switches for w2 in switches])

plot_generator = graph_plot_generator([replicated_switch], vertex_size=6000,
    font_size=35, edge_thickness=5, border_style_default="")

\end{sagesilent}

\begin{figure}
\centering
\sageplot[width=4cm][pdf]{
next(plot_generator)
}
\caption{Graph corresponding to the replicated switch example.}
\label{fig:replicated-switch}
\end{figure}


\begin{sagesilent}

def exists_connected_component_spanning_slaves(graph):
    connected_components = graph.connected_components()
    for cc in connected_components:
        if set(slaves).issubset(cc):
            return True
    return False

def is_connected_graph_spanning_slaves(graph):
    if graph.is_connected() and set(slaves).issubset(graph.vertices()):
        return True
    else:
        return False

REPLICATED_SWITCH_EDGE_FAILURE_RATE = 1/100
REPLICATED_SWITCH_VERTEX_FAILURE_RATE = 2/100

edge_failure_rates = {edge[:2]: REPLICATED_SWITCH_EDGE_FAILURE_RATE
    for edge in replicated_switch.edges()}
vertex_failure_rates = {vertex: REPLICATED_SWITCH_VERTEX_FAILURE_RATE
    for vertex in replicated_switch.vertices()}
failure_rates = dict(edge_failure_rates.items() + vertex_failure_rates.items())

digraph, transition_rate_matrix = pm.generate_ctmc(replicated_switch,
    failure_rates)

transitions_file_name, states_file_name = pm.create_prism_files(
    transition_rate_matrix, 'prism/replicated_star')

non_faulty_states = pm.filter_graphs(digraph,
    exists_connected_component_spanning_slaves)
pm.prism_properties(non_faulty_states, 'prism/replicated_star.pctl')

now_str = datetime.datetime.now().isoformat()
results_dir_name = "results/replicated_star_{}/".format(now_str)
os.makedirs(results_dir_name)
initial_states = pm.filter_graphs(digraph,
    is_connected_graph_spanning_slaves)
pm.run_experiments(transitions_file_name, states_file_name,
    'prism/replicated_star.pctl', results_dir_name,
    initial_states, EXPERIMENT_DURATION)

results_plot = pm.plot_results(results_dir_name, EXPERIMENT_DURATION)

subgraphs = []
vertex_numbers = []
non_faulty_subgraphs = []
for v in digraph.vertices():
    subgraphs.append(digraph.get_vertex(v))
    vertex_numbers.append(v)
    if v in non_faulty_states:
        non_faulty_subgraphs.append(digraph.get_vertex(v))

subgraph_vertex_plots = graph_plot_generator(subgraphs,
    graphs_to_highlight=non_faulty_subgraphs)

non_faulty_vertex_plots = graph_plot_generator(non_faulty_subgraphs,
    graphs_to_highlight=non_faulty_subgraphs)

induced_subdigraph_plots = induced_subdigraph_plot_generator(digraph)

initial_states_subgraphs = []
for v in digraph.vertices():
    if v in initial_states:
        initial_states_subgraphs.append(digraph.get_vertex(v))

initial_states_plots = graph_plot_generator(initial_states_subgraphs,
    graphs_to_highlight=initial_states_subgraphs)


\end{sagesilent}

Figure~\ref{fig:replicated-switch-digraph} shows the digraph whose vertices are
the subgraphs of Figure~\ref{fig:replicated-switch}. Since this digraph is too
dense to be readable, Figure~\ref{fig:replicated_switch_digraphs_1} shows the
subdigraphs induced by each of the digraph's vertices together with its
out-neighbors. The subgraphs of the replicated switch graph corresponding to
each vertex of the digraph are shown in
Figure~\ref{fig:replicated_switch_subgraphs_1}.

We consider a vertex of the digraph non-faulty if it contains a \emph{connected
component} that spans all the stations. Such vertices have been highlighted in
Figure~\ref{fig:replicated_switch_subgraphs_1} by a solid border.


\begin{figure}
\centering
\sageplot[width=\columnwidth][pdf]{
digraph.plot(layout="circular")
}
\caption{Digraph corresponding to the replicated switch.}
\label{fig:replicated-switch-digraph}
\end{figure}

\sagestr{generate_latex_figures(induced_subdigraph_plots, vertex_numbers,
    "replicated_switch_digraphs_",
    'Induced subdigraphs.',
    'fig:replicated_switch_digraphs_', numrows=7)}

\sagestr{generate_latex_figures(subgraph_vertex_plots, vertex_numbers,
    "replicated_switch_vertices_",
    'Vertices of the digraph corresponding to the replicated switch.',
    'fig:replicated_switch_subgraphs_')}

\sagestr{generate_latex_figures(initial_states_plots, initial_states,
    "replicated_switch_designs_",
    'Vertices of the digraph considered as potential designs for' +
    ' the replicated switch.',
    'fig:replicated_switch_designs_')}



Figure~\ref{fig:replicated-switch-results} shows the reliability achievable
with each subgraph when we have a failure rate of
$\sage{REPLICATED_SWITCH_VERTEX_FAILURE_RATE}$ for all stations and switches
and a failure rate of $\sage{REPLICATED_SWITCH_EDGE_FAILURE_RATE}$ for all links
and interlinks.


\begin{figure}
\centering
\sageplot[width=\columnwidth][pdf]{
results_plot
}
\caption{Results for the replicated switch.}
\label{fig:replicated-switch-results}
\end{figure}

\subsection{Master/slave star-based networks}


\begin{itemize}

\item The network is based on master/slave communication.

\end{itemize}





\subsubsection{Objective}

Our goal is the following:

\begin{itemize}

\item Determine which of all potential architectures for a master/slave
star-based network has the highest reliability considering only permanent
faults.

\end{itemize}




\subsubsection{Things that may vary between the potential architectures}

\begin{itemize}

\item Location of the masters: within nodes or switches.

\item Number of masters.

\item Number of switches.

\item How switches are interconnected by means of links.

\item To which switches the slaves are connected.

\item To which switches the master nodes are connected.

\end{itemize}






\begin{sagesilent}

SLAVE_COUNT = 3
SWITCH_COUNT = 2
MASTER_NODE_COUNT = 2
EMBEDDED_MASTER_COUNT = SWITCH_COUNT

slaves = ["$s_{}$".format(i) for i in range(1, SLAVE_COUNT + 1)]
switches = ["$w_{}$".format(i) for i in range(1, SWITCH_COUNT + 1)]
master_nodes = ["$m_{}$".format(i) for i in range(1, MASTER_NODE_COUNT + 1)]
embedded_masters = ["$b_{}$".format(i) for i in
    range(1, EMBEDDED_MASTER_COUNT + 1)]

master_slave_star = Graph()
# add interlinks
master_slave_star.add_edges(CartesianProduct(switches, switches))
# add slave links
master_slave_star.add_edges(CartesianProduct(slaves, switches))
# add master links
master_slave_star.add_edges(CartesianProduct(master_nodes, switches))
# add embedded links
master_slave_star.add_edges(zip(embedded_masters, switches))

plot_generator = graph_plot_generator([master_slave_star], vertex_size=6000,
    font_size=35, layout="spring", edge_thickness=5, border_style_default="")

\end{sagesilent}

\begin{figure}
\centering
\sageplot[width=4cm][pdf]{
next(plot_generator)
}
\caption{Graph corresponding to the master/slave star-based network example.}
\label{fig:master-slave-star}
\end{figure}


%\sagetexpause

\begin{sagesilent}

def is_non_faulty_master_slave_subgraph(graph):
    """
    We consider the graph as non-faulty if it contains a connected component
    that has no slave that is a cut vertex and if it spans all slaves and at
    least one master.
    """
    connected_components = graph.connected_components()
    for cc in connected_components:
        contains_all_slaves = set(slaves).issubset(cc)
        contains_at_least_1_master = (set(cc).intersection(set(master_nodes)) or
            set(cc).intersection(set(embedded_masters)))
        if contains_all_slaves and contains_at_least_1_master:
            cc_subgraph = graph.subgraph(cc)
            for s in slaves:
                if cc_subgraph.is_cut_vertex(s):
                    return False
            return True
    return False

def is_master_slave_design(graph):
    """
    We consider the graph as a potential design for the master slave star if
    it is connected and non-faulty.
    """
    if graph.is_connected():
        v = graph.vertices()
        contains_all_slaves = set(slaves).issubset(v)
        contains_at_least_1_master = (set(v).intersection(set(master_nodes)) or
            set(v).intersection(set(embedded_masters)))
        if contains_all_slaves and contains_at_least_1_master:
            for s in slaves:
                if graph.is_cut_vertex(s):
                    return False
            return True
    return False

FAILURE_RATE = {
    'slave': float("3e-6"),
    'switch': float("6e-6"),
    'master node': float("3e-6"),
    'embedded master': float("3e-6"),
    'slave link': float("3.6e-6"),
    'interlink': float("3.6e-6"),
    'master link': float("3.6e-6"),
    'embedded link': 0
}

\end{sagesilent}

\sagestr{generate_latex_table(FAILURE_RATE,
    "Failure rates for the master/slave star-based network.",
    "tab:master_slave_star_failure_rates")}

%\sagetexpause

\begin{sagesilent}

vertex_failure_rates = [(vertex, FAILURE_RATE['slave'])
    for vertex in slaves]
vertex_failure_rates += [(vertex, FAILURE_RATE['switch'])
    for vertex in switches]
vertex_failure_rates += [(vertex, FAILURE_RATE['master node'])
    for vertex in master_nodes]
vertex_failure_rates += [(vertex, FAILURE_RATE['embedded master'])
    for vertex in embedded_masters]

edge_failure_rates = [(edge[:2], FAILURE_RATE['slave link'])
    for edge in master_slave_star.edges() if edge[0] in slaves]
edge_failure_rates += [(edge[:2], FAILURE_RATE['interlink'])
    for edge in master_slave_star.edges() if edge[0] in switches and
    edge[1] in switches]
edge_failure_rates += [(edge[:2], FAILURE_RATE['master link'])
    for edge in master_slave_star.edges() if edge[0] in master_nodes]
edge_failure_rates += [(edge[:2], FAILURE_RATE['embedded link'])
    for edge in master_slave_star.edges() if edge[0] in embedded_masters]

failure_rates = dict(edge_failure_rates + vertex_failure_rates)

digraph, transition_rate_matrix = pm.generate_ctmc(master_slave_star,
    failure_rates)
transitions_file_name, states_file_name = pm.create_prism_files(
    transition_rate_matrix, 'prism/master_slave_star')
non_faulty_states = pm.filter_graphs(digraph,
    is_non_faulty_master_slave_subgraph)
pm.prism_properties(non_faulty_states, 'prism/master_slave_star.pctl')

now_str = datetime.datetime.now().isoformat()
results_dir_name = "results/master_slave_star_{}/".format(now_str)
os.makedirs(results_dir_name)
initial_states = pm.filter_graphs(digraph, is_master_slave_design)

pm.run_experiments(transitions_file_name, states_file_name,
    'prism/master_slave_star.pctl', results_dir_name,
    initial_states, EXPERIMENT_DURATION)

results_plot_all = pm.plot_results(results_dir_name, EXPERIMENT_DURATION)
results_plot_5_best = pm.plot_results(results_dir_name, EXPERIMENT_DURATION,
    max_curves=5)
results_plot_10_best = pm.plot_results(results_dir_name, EXPERIMENT_DURATION,
    max_curves=10)

initial_states_subgraphs = []
for v in digraph.vertices():
    if v in initial_states:
        initial_states_subgraphs.append(digraph.get_vertex(v))

initial_states_plots = graph_plot_generator(initial_states_subgraphs,
    graphs_to_highlight=initial_states_subgraphs)

\end{sagesilent}

\sagestr{generate_latex_figures(initial_states_plots, initial_states,
    "master_slave_star_non_faulty_and_connected_",
    'Non-faulty and connected designs corresponding to the master/slave' +
    ' star-based network.',
    'fig:master_slave_star_non_faulty_and_connected_')}

\begin{figure}
\centering
\sageplot[width=\columnwidth][pdf]{
results_plot_all
}
\caption{All results for the master/slave replicated star.}
\label{fig:master_slave_star_results_all}
\end{figure}

\begin{figure}
\centering
\sageplot[width=\columnwidth][pdf]{
results_plot_5_best
}
\caption{Five best results for the master/slave replicated star.}
\label{fig:master_slave_star_results_5_best}
\end{figure}

\begin{figure}
\centering
\sageplot[width=\columnwidth][pdf]{
results_plot_10_best
}
\caption{Ten best results for the master/slave replicated star.}
\label{fig:master_slave_star_results_10_best}
\end{figure}





\section*{Acknowledgements}

This work was supported by project DPI2011-22992 and grant BES-2012-052040
(Spanish \emph{Ministerio de econom\'ia y competividad}), and by FEDER funding.






\bibliographystyle{abbrv}
\bibliography{inc-3}


\end{document}

% vim: tabstop=8 expandtab shiftwidth=4 softtabstop=4
